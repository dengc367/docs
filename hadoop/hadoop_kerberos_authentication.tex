\documentclass[a4paper,12pt]{article}
\usepackage{CJK}

\begin{document}
\begin{CJK}{UTF8}{gbsn}

\section{Introduction}

    Kerberos Security Management Mechanism is new to hadoop 1.0 and Cloudera Distributed Hadoop.
在hadoop 1.0和CDH3中新加上了kerberos安全机制来控制hadoop集群的访问权限。在之前的版本,都没有kerberos安全论证之说,默认集群是可靠的,值得信赖的。在mr和hdfs交互时都不需要认证。没有强有力的安全保障会使一些恶意用户恶意提交作业,篡改jobtracker状态,修改hdfs文件,甚至直接伪装成合法用户去修改Hadoop集群的状态来控制hadoop集群。当然任何系统都没有绝对安全之说,只有安全的级别高低之分。
Kerberos可以将认证的密钥放到可靠的节点上,当集群运行时,集群内的节点使用密钥来认证。只有认证通过的节点才能加入到集群中来,流氓节点由于没有密钥信息而通不过验证,无法与集群内部的节点进行通信。防止了集群的恶意修改,保障了hadoop集群的安全运行。

\section{Hadoop Kerberos Configuration}

\subsection{core-site.xml}

在core-site.xml中配置如下参数。  

\begin{verbatim}
<property>
  <name>hadoop.security.authorization</name>
  <value>true</value>
  <description>Is service-level authorization enabled?</description>
</property>
<property>
  <name>hadoop.rpc.protection</name>
  <value>privacy</value>
  <description>Possible values are authentication (no integrity, privacy)
  </description>
</property>
<property>
  <name>hadoop.security.authentication</name>
  <value>kerberos</value>
  <description>Possible values are simple (no authentication), and kerberos
  </description>
</property>
\end{verbatim}

\subsection{hdfs-site.xml}

在hdfs-site.xml中配置如下参数。
 
\begin{verbatim}
<property>
 <name>dfs.https.address</name>
 <value>namenodeHost:50470</value>
</property>
<property>
 <name>dfs.https.port</name>
 <value>50470</value>
</property>
<property>
 <name>dfs.block.access.token.enable</name>
 <value>true</value>
</property>
<property>
 <name>dfs.namenode.keytab.file</name>
 <value>/home/hadoop/hadoop/conf/hadoop.keytab</value>
</property>
<property>
 <name>dfs.namenode.kerberos.principal</name>
 <value>hadoop/_HOST@<KERBEROS_DOMAIN_NAME></value>
</property>
<property>
 <name>dfs.namenode.kerberos.https.principal</name>
 <value>host/_HOST@<KERBEROS_DOMAIN_NAME></value>
</property>
<property>
  <name>dfs.datanode.data.dir.perm</name>
  <value>755</value>        
  <description>Permissions for the directories on on the local filesystem where
  the DFS data node store its blocks. The permissions can either be octal or
  symbolic.
  </description>
 </property>
 <property>
  <name>dfs.datanode.address</name>
  <value>0.0.0.0:1004</value>
 </property>
 <property>
  <name>dfs.datanode.http.address</name>
  <value>0.0.0.0:1006</value>
 </property>
 <property>
  <name>dfs.datanode.keytab.file</name>
  <value>/home/hadoop/hadoop/conf/hadoop.keytab</value> 
</property>
<property>
  <name>dfs.datanode.kerberos.principal</name>
  <value>hadoop/_HOST@<KERBEROS_DOMAIN_NAME></value>
</property>
<property>
  <name>dfs.datanode.kerberos.https.principal</name>
  <value>host/_HOST@<KERBEROS_DOMAIN_NAME></value>
</property>
\end{verbatim}
\subsection{mapred-site.xml}
在mapred-site.xml中配置如下参数。
\begin{verbatim}
<property>
  <name>mapreduce.jobtracker.kerberos.principal</name>
  <value>hadoop/_HOST@<KERBEROS_DOMAIN_NAME></value>
</property>
<property>
  <name>mapreduce.jobtracker.kerberos.https.principal</name>
  <value>host/_HOST@<KERBEROS_DOMAIN_NAME></value>
</property>
<property>
  <name>mapreduce.jobtracker.keytab.file</name>
  <value>/home/hadoop/hadoop/conf/hadoop.keytab</value>
</property>
<property>
  <name>mapreduce.tasktracker.kerberos.principal</name>
  <value>hadoop/_HOST@<KERBEROS_DOMAIN_NAME></value>
</property>
<property>
  <name>mapreduce.tasktracker.kerberos.https.principal</name>
  <value>host/_HOST@<KERBEROS_DOMAIN_NAME></value>
</property>
<property>
  <name>mapreduce.tasktracker.keytab.file</name>
  <value>/home/hadoop/hadoop/conf/hadoop.keytab</value>
</property>
<property>
  <name>mapred.task.tracker.task-controller</name>
  <value>org.apache.hadoop.mapred.DefaultTaskController</value>
</property>
<property>
  <name>mapreduce.tasktracker.group</name>
  <value>hadoop</value>
</property>
\end{verbatim}
\subsection{hadoop Startup Command}

\begin{verbatim}
hadoop@_HOST:~\$ ~hadoop/hadoop/bin/hadoop-daemon.sh start namenode
root@_HOST:~\# ~hadoop/hadoop/bin/hadoop-daemon.sh start namenode
hadoop@_HOST:~\$ ~hadoop/hadoop/bin/hadoop-daemon.sh start jobtracker
hadoop@_HOST:~\$ ~hadoop/hadoop/bin/hadoop-daemon.sh start tasktracker
\end{verbatim}

\subsection{Hadoop 1.0.3 fair-scheduler Kerberos Authentication Bug(when Job Submited)} 

这个bug解决得比较曲折。在官方的bug网站上已经有了相应的bug fix。找出了最终的原因是ExecutorService的一种机制:ExecutorService可以产生若干个线程,但是不会同时生成(也就是说按需生成),因此在应该产生的时候就没有kerberos的相应的认证信息了。
MAPREDUCE-4451 -- fairscheduler fail to init job with kerberos authentication configured
\section{Secure HBase Kerberos Configuration}

\subsection{hbase-site.xml}
\begin{verbatim}
<property>
        <name>hbase.regionserver.kerberos.principal</name>
        <value>hbase/_HOST@<KERBEROS_DOMAIN_NAME></value>
</property>
<property>
        <name>hbase.regionserver.keytab.file</name>
        <value>/home/hbase/hbase/conf/hdfs.keytab</value>
</property>
<property>
        <name>hbase.master.kerberos.principal</name>
        <value>hbase/_HOST@<KERBEROS_DOMAIN_NAME></value>
</property>
<property>
        <name>hbase.master.keytab.file</name>
        <value>/home/hbase/hbase/conf/hdfs.keytab</value>
</property>
<property>
        <name>hbase.security.authentication</name>
        <value>kerberos</value>
</property>
<property>
        <name>hbase.security.authorization</name>
        <value>true</value>
</property>
<property>
        <name>hbase.rpc.engine</name>
        <value>org.apache.hadoop.hbase.ipc.SecureRpcEngine</value>
</property>
<property>
        <name>hbase.rpc.protection</name>
        <value>privacy</value>
</property>
\end{verbatim}

\subsection{Secure HBase Startup Command}
\begin{verbatim}
hbase@_HOST:~\$ ~hbase/hbase/bin/hbase-daemon.sh start master
hbase@_HOST:~\$ ~hbase/hbase/bin/hbase-daemon.sh start regionserver
\end{verbatim}

\subsection{HBase Security Coprocessor Configuration}
\begin{verbatim}
<property>
  <name>hbase.coprocessor.master.classes</name>
  <value>org.apache.hadoop.hbase.security.access.AccessController</value>
</property>
<property>
  <name>hbase.coprocessor.region.classes</name>
  <value>org.apache.hadoop.hbase.security.token.TokenProvider,org.apache.hadoop.hbase.security.access.AccessController</value>
</property>
\end{verbatim}

\subsection{HTable Security Command} 

\begin{verbatim}
user_permission
grant
revoke
\end{verbatim}


\end{CJK}
\end{document}
